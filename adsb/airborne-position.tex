\section{Airborne Positions}\label{airborne-positions}

An aircraft airborne position message has downlink format \texttt{17} (or \texttt{18}) with type code from \texttt{9} to \texttt{18}

Messages are composed as shown in following Table \ref{tb:adsb-pos-bits}

\begin{table}[!ht]
\centering
\caption{Airborne position message bits explained}
\label{tb:adsb-pos-bits}
\begin{tabular}{@{}lllll@{}}
\toprule
Data Bits & MSG Bits & N-bit & Abbr    & Content                 \\ \midrule
33 - 37   & 1 - 5    & 5     & TC      & Type code               \\
38 - 39   & 6 - 7    & 2     & SS      & Surveillance status     \\
40        & 8        & 1     & NICsb   & NIC supplement-B        \\
41 - 52   & 9 - 20   & 12    & ALT     & Altitude                \\
53        & 21       & 1     & T       & Time                    \\
54        & 22       & 1     & F       & CPR odd/even frame flag \\
55 - 71   & 23 - 39  & 17    & LAT-CPR & Latitude in CPR format  \\
72 - 88   & 40 - 56  & 17    & LON-CPR & Longitude in CPR format \\ \bottomrule
\end{tabular}
\end{table}

Two types of the position messages (odd and even frames) are broadcast alternately. There are two different ways to decode an airborne position based on these messages:

\begin{enumerate}
\def\labelenumi{\arabic{enumi}.}

\item
  Unknown position, using both types of messages (aka globally unambiguous position)
\item
  Knowing previous position, using only one message (aka locally unambiguous position)
\end{enumerate}

Note: The definition of functions \texttt{NL(lat)}, \texttt{floor(x)}, and \texttt{mod(x,y)} are described in the CPR chapter.

\subsection{Globally unambiguous position (decoding with two messages)}\label{globally-unambiguous-position-decoding-with-two-messages}

\subsubsection{odd or even message?}\label{odd-or-even-message}

For each frame, bit 54 determines whether it is an odd or even frame:

\begin{verbatim}
0 -> Even frame
1 -> Odd frame
\end{verbatim}

For example, the two following messages are received:

\begin{verbatim}
8D40621D58C382D690C8AC2863A7
8D40621D58C386435CC412692AD6

|    | ICAO24 |      DATA      |  CRC   |
|----|--------|----------------|--------|
| 8D | 40621D | 58C382D690C8AC | 2863A7 |
| 8D | 40621D | 58C386435CC412 | 692AD6 |

\end{verbatim}

The payload data in binary format:

\begin{verbatim}
| DATA                                                                       |
|============================================================================|
| TC    | ... | ALT          | T | F | CPR-LAT           | CPR-LON           |
|-------|-----|--------------|---|---|-------------------|-------------------|
| 01011 | 000 | 110000111000 | 0 | 0 | 10110101101001000 | 01100100010101100 |
| 01011 | 000 | 110000111000 | 0 | 1 | 10010000110101110 | 01100010000010010 |
\end{verbatim}

In both messages we can find \texttt{DF=17} and \texttt{TC=11}, with the same ICAO24 address \texttt{40621D}. So, those two frames are valid for decoding the positions of this aircraft. Assume the first message is the newest message received.

\subsubsection{The CPR representation of
coordinates}\label{the-cpr-representation-of-coordinates}

\begin{verbatim}
| F | CPR Latitude      | CPR Longitude     |
|---|-------------------|-------------------|
| 0 | 10110101101001000 | 01100100010101100 |  -> newest
| 1 | 10010000110101110 | 01100010000010010 |
|---|-------------------|-------------------|

In decimal:

|---|-------------------|-------------------|
| 0 | 93000             | 51372             |
| 1 | 74158             | 50194             |
|---|-------------------|-------------------|

CPR_LAT_EVEN: 93000 / 131072 -> 0.7095
CPR_LON_EVEN: 51372 / 131072 -> 0.3919
CPR_LAT_ODD:  74158 / 131072 -> 0.5658
CPR_LON_ODD:  50194 / 131072 -> 0.3829
\end{verbatim}

Since CPR latitude and longitude are encoded in 17 bits, 131072 ($2^{17}$) is the maximum value.

\subsubsection{Calculate the latitude index j}\label{calculate-the-latitude-index-j}

Use the following equation:

\begin{equation}
  j = floor \left( 59 \cdot \mathrm{Lat}_\mathrm{cprEven} - 60 \cdot \mathrm{Lat}_\mathrm{cprOdd} + \frac{1}{2}  \right)
\end{equation}

\noindent where $j$ is set 8.

\subsubsection{Calculate latitude}\label{calculate-latitude}

First, two constants will be used:

\begin{equation}
  \begin{split}
    \mathrm{dLat}_\mathrm{even} &= \frac{360}{4 \cdot NZ} = \frac{360}{60} \\
    \mathrm{dLat}_\mathrm{odd} &= \frac{360}{4 \cdot NZ - 1}  = \frac{360}{59}
  \end{split}
\end{equation}

Then we can use the following equations to compute the relative latitudes:

\begin{equation}
  \begin{split}
    \mathrm{Lat}_\mathrm{even} &= \mathrm{dLat}_\mathrm{even} \cdot [mod(j, 60) + \mathrm{Lat}_\mathrm{cprEven}] \\
    \mathrm{Lat}_\mathrm{odd} &= \mathrm{dLat}_\mathrm{odd} \cdot [mod(j, 59) + \mathrm{Lat}_\mathrm{cprOdd}]
  \end{split}
\end{equation}

For the southern hemisphere, values will fall from 270 to 360 degrees. We need to make sure the latitude is within the range
\texttt{{[}-90,\ +90{]}}:

\begin{equation}
  \begin{split}
    \mathrm{Lat}_\mathrm{even} &= \mathrm{Lat}_\mathrm{even} - 360  \quad \text{if } (\mathrm{Lat}_\mathrm{even} \geq 270) \\
    \mathrm{Lat}_\mathrm{odd} &= \mathrm{Lat}_\mathrm{odd} - 360  \quad \text{if } (\mathrm{Lat}_\mathrm{odd} \geq 270)
  \end{split}
\end{equation}

Final latitude is chosen depending on the time stamp of the frames, the newest one, is used:

\begin{equation}
  \mathrm{Lat} =
  \begin{cases}
   \mathrm{Lat}_\mathrm{even}     & \text{if } (T_\mathrm{even} \geq T_\mathrm{odd}) \\
   \mathrm{Lat}_\mathrm{odd}     & \text{else}
  \end{cases}
\end{equation}

In the example:

\begin{verbatim}
Lat_EVEN = 52.25720214843750
Lat_ODD  = 52.26578017412606
Lat = Lat_EVEN = 52.25720
\end{verbatim}

\subsubsection{Check the latitude zone
consistency}\label{check-the-latitude-zone-consistency}

Compute \texttt{NL(Lat\_E)} and \texttt{NL(Lat\_O)}. If not the same, two positions are located at different latitude zones. Computation of a global longitude is not possible. Exit the calculation and wait for new messages. If two values are the same, we proceed to longitude calculation.

\subsubsection{Calculate longitude}\label{calculate-longitude}

If the even frame comes latest \texttt{T\_EVEN\ \textgreater{}\ T\_ODD}:

\begin{equation}
  \begin{split}
    ni &= max \left( NL(\mathrm{Lat}_\mathrm{even}), 1 \right) \\
    \mathrm{dLon} &= \frac{360}{ni} \\
    m &= floor \left\{ Lon_\mathrm{cprEven} \cdot [NL(\mathrm{Lat}_\mathrm{even})-1] - Lon_\mathrm{cprOdd} \cdot NL(\mathrm{Lat}_\mathrm{even}) + \frac{1}{2}  \right\} \\
    \mathrm{Lon} &= \mathrm{dLon} \cdot \left( mod(m, ni) + Lon_\mathrm{cprEven} \right)
  \end{split}
\end{equation}

In case where the odd frame comes latest
\texttt{T\_EVEN\ \textless{}\ T\_ODD}:

\begin{equation}
  \begin{split}
    ni &= max \left( NL(\mathrm{Lat}_\mathrm{odd})-1, 1 \right) \\
    \mathrm{dLon} &= \frac{360}{ni} \\
    m &= floor \left\{ Lon_\mathrm{cprEven} \cdot [NL(\mathrm{Lat}_\mathrm{odd})-1] - Lon_\mathrm{cprOdd} \cdot NL(\mathrm{Lat}_\mathrm{odd}) + \frac{1}{2}  \right\} \\
    \mathrm{Lon} &= \mathrm{dLon} \cdot \left( mod(m, ni) + Lon_\mathrm{cprOdd} \right)
  \end{split}
\end{equation}

if the result is larger than 180 degrees:

\begin{equation}
  \mathrm{Lon} = \mathrm{Lon} - 360  \quad \text{if } (\mathrm{Lon} \geq 180)
\end{equation}

In the example:

\begin{verbatim}
Lon:  3.91937
\end{verbatim}

Here is a Python implementation:
\url{https://github.com/junzis/pyModeS/blob/faf4313/pyModeS/adsb.py\#L166}

\subsubsection{Calculate altitude}\label{calculate-altitude}

The altitude of the aircraft is much easier to compute from the data frame. The bits in the altitude field (either odd or even frame) are as follows:

\begin{verbatim}
1100001 1 1000
        ^
       Q-bit
\end{verbatim}

This Q-bit (bit 48) indicates whether the altitude is encoded in multiples of 25 or 100 ft (0: 100 ft, 1: 25 ft).

For Q = 1, we can calculate the altitude as follows:

First, remove the Q-bit :

\begin{verbatim}
N = 1100001 1000 => 1560 (in decimal)
\end{verbatim}

The final altitude value will be:

\begin{equation}
  Alt = N \cdot 25 - 1000 \quad \text{(ft.)}
\end{equation}

In this example, the altitude at which the aircraft is flying is:

\begin{verbatim}
1560 * 25 - 1000 = 38000 ft.
\end{verbatim}

Note that the altitude has the accuracy of +/- 25 ft when the Q-bit is 1, and the value can represent altitudes from -1000 to +50175 ft.

\subsubsection{The final position}\label{the-final-position}

Finally, we have all three components (latitude/longitude/altitude) of the aircraft position:

\begin{verbatim}
LAT: 52.25720 (degrees N)
LON:  3.91937 (degrees E)
ALT:    38000 ft
\end{verbatim}

\subsection{Locally unambiguous position (decoding with one
message)}\label{locally-unambiguous-position-decoding-with-one-message}

This method gives the possibility of decoding aircraft using only one message knowing a reference position. This method computes the latitude index (j) and the longitude index (m) based on such reference, and can be used with either even/odd type of messages.

\subsubsection{The reference position}\label{the-reference-position}

The reference position should be close to the actual position (eg. position of aircraft previously decoded, or the location of ADS-B antenna), and must be \textbf{within a 180 NM} range.

\subsubsection{Calculate dLat}\label{calculate-dlat}

\begin{equation}
  dLat =
  \begin{cases}
   \frac{360}{4 \cdot NZ} = \frac{360}{60}          & \text{for even message}  \\
   \frac{360}{4 \cdot NZ - 1}  = \frac{360}{59}     & \text{for odd message}
  \end{cases}
\end{equation}

\subsubsection{Calculate the latitude index j} \label{calculate-the-latitude-index-j-1}

\begin{equation}
  j = floor \left (\frac{\mathrm{Lat}_{ref}}{dLat} \right) + floor \left( \frac{mod(\mathrm{Lat}_{ref}, dLat)}{dLat}  - \mathrm{Lat}_\mathrm{cpr}  + \frac{1}{2} \right)
\end{equation}

\subsubsection{Calculate latitude}\label{calculate-latitude-1}

\begin{equation}
  \mathrm{Lat} = dLat \cdot (j + \mathrm{Lat}_\mathrm{cpr})
\end{equation}


\subsubsection{Calculate dLon}\label{calculate-dlon}

For even message:

\begin{equation}
  \mathrm{dLon} =
  \begin{cases}
   \frac{360}{NL(Lat)}    & \text{if } NL(Lat) > 0  \\
   360                    & \text{if } NL(Lat) = 0
  \end{cases}
\end{equation}

For odd message:
\begin{equation}
  \mathrm{dLon} =
  \begin{cases}
   \frac{360}{NL(Lat)-1}    & \text{if } NL(Lat) - 1 > 0  \\
   360                    & \text{if } NL(Lat) - 1 = 0
  \end{cases}
\end{equation}


\subsubsection{Calculate longitude index
m}\label{calculate-longitude-index-m}

\begin{equation}
  m = floor \left( \frac{Lon_{ref}}{\mathrm{dLon}} \right) + floor \left( \frac{mod(Lon_{ref}, \mathrm{dLon})}{\mathrm{dLon}}  - Lon_\mathrm{cpr}  + \frac{1}{2}  \right)
\end{equation}

\subsubsection{Calculate longitude}\label{calculate-longitude-1}

\begin{equation}
  Lon = \mathrm{dLon} \cdot (m + Lon_\mathrm{cpr})
\end{equation}

\subsubsection{Example}\label{example}

For the same example message:

\begin{verbatim}
8D40621D58C382D690C8AC2863A7

Reference position:
  LAT: 52.258
  LON:  3.918
\end{verbatim}

The structure of the message is:

\begin{verbatim}
8D40621D58C382D690C8AC2863A7

|    | ICAO24 |      DATA      |  CRC   |
|----|--------|----------------|--------|
| 8D | 40621D | 58C382D690C8AC | 2863A7 |


Data in binary:

| DATA                                                                       |
|============================================================================|
| TC    | ... | ALT          | T | F | CPR-LAT           | CPR-LON           |
|-------|-----|--------------|---|---|-------------------|-------------------|
| 01011 | 000 | 110000111000 | 0 | 0 | 10110101101001000 | 01100100010101100 |


CPR representation:

| F | CPR Latitude      | CPR Longitude     |
|---|-------------------|-------------------|
| 0 | 10110101101001000 | 01100100010101100 |
|---|-------------------|-------------------|
|   | 93000 / 131072    | 51372 / 131072    |
|   | 0.7095            | 0.3919            |
|---|-------------------|-------------------|

d_lat:  6
j:      8
lat:    52.25720
m:      0
d_lon:  10
lon:    3.91937
\end{verbatim}
